\section{Research Gap}
\large
One potential research gap in this area is the exploration of the practical applications of using Merkle trees with the least verification time for data integrity verification. While the use of Merkle trees for data integrity verification is not a new concept, the focus on minimizing verification time could be a novel contribution. Minimizing verification time can enhance the efficiency and scalability of the data integrity audit scheme, especially in large-scale and complex systems, which require frequent and fast data verification. Therefore, there is a need to investigate the potential applications of this approach in various domains and determine how it can be applied effectively.

In our research, we focused on the use of Merkle trees with the least verification time for data integrity verification. We aimed to address the limitations of traditional Merkle tree-based schemes, which may require extensive computation and verification time, particularly in large-scale and complex systems. Our approach aimed to enhance the efficiency and scalability of the data integrity audit scheme by minimizing the verification time while ensuring the security and accuracy of data verification.

We developed a novel algorithm that utilizes a binary tree structure to reduce the time required to verify data integrity using Merkle trees. Our algorithm partitions the data into sub-blocks and computes their corresponding Merkle trees. The sub-blocks' Merkle roots are then combined to form the parent node of the binary tree structure, which reduces the number of Merkle roots that need to be computed during verification. This approach significantly reduces the verification time while ensuring the security and accuracy of data verification.

Our research aimed to contribute to the development of more efficient and scalable methods for verifying data integrity in large-scale and complex systems. We conducted experiments to compare the performance of our approach with traditional Merkle tree-based schemes, and the results demonstrated the effectiveness of our approach in reducing the verification time without compromising the security and accuracy of data verification.