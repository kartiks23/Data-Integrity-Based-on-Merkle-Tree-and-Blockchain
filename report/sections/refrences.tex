\section{Refrences}
\large

[1] Jianbing Ni, Kuan Zhang, Yong Yu, Tingting Yang: Identity-Based Provable Data Possession From RSA Assumption for Secure Cloud Storage. IEEE Trans. Dependable Secur. Comput. 19(3): 1753-1769 (2022).

[2] Yang Yang, Yanjiao Chen, Fei Chen, Jing Chen: An Efficient Identity-Based Provable Data Possession Protocol With Compressed Cloud Storage. IEEE Trans. Inf. Forensics Secur. 17: 1359-1371 (2022).

[3] Genqing Bian, Jinyong Chang, “Certificateless Provable Data Possession Protocol for the Multiple Copies and Clouds Case,” IEEE Access 8: 102958-102970 (2020).

[4] Françoise Levy-dit-Vehel, Maxime Roméas:A Framework for the Design of Secure and Efficient Proofs of Retrievability. IACR Cryptol. ePrint Arch. 2022: 64 (2022).

[5] Xiao Zhang, Shengli Liu, Shuai Han,“Proofs of retrievability from linearly homomorphic structure-preserving Tag natures,” Int. J. Inf. Comput. Secur. 11(2): 178-202 (2019).

[6] Binanda Sengupta, Sushmita Ruj,“Efficient Proofs of Retrievability with Public Verifiability for Dynamic Cloud Storage,”IEEE Trans. Cloud Comput. 8(1): 138-151 (2020).

[7] L. Zhu, Y. Wu, K. Gai, and K.-K. R. Choo, “Controllable and trustworthy blockchain-based cloud data management,” Future Generation Computer Systems, vol. 91, pp. 527–535, 2019.

[8] L. Min, L. You and P. Fei, “Integrity verification for multiple data copies in cloud storage based on quad merkle tree” IEEE International journal of bifurcation and Chaos, vol 27, no 4, 2017.
